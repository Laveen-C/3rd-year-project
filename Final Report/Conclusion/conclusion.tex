\chapter{Conclusion}

\section{Project Outcomes}
Overall, the project has been mostly successful. On the literature review side, we have covered all of the relevant background knowledge on Dyck words and their combinatorial properties in a rigorous yet intuitive manner. We've then built upon this foundational knowledge when reviewing the "Re-Pairing Brackets" paper. The content has been covered in a manner such that we not only made claims and observations, but gave well-founded reasoning for each of them. Although some of the results were well known, they were derived in a 'first-principles' like manner. We also discussed the larger scale impact and links between the re-pairing game in other areas of theoretical computer science.

The software developed has been largely successful as well. Most functionality is present, with a lot of care put into handling errors, edge cases, and sequences of interactions in an appropriate manner. The software implements both the Simple strategy from literature, and the Greedy algorithm as an improvement. The manual experimentation option is also robust, and it is the author's belief that the web application can be used to visualise and analyse the software with high efficiency and simplicity. It's high modularity for adding and removing strategies makes it well suited for future work on the topic.

\section{Limitations}
Whilst there was much accomplished, there are some shortcomings with the project as well. The non-simple recursive strategy from the literature was not implemented, and there has not been much progress in the way of an efficient exhaustive width search function. These were due to a struggle with time constraints 

\section{Further Work}
The following possible extensions would greatly enhance the scope of the project and provide potentially novel discoveries if pursused:

\begin{enumerate}
    \item Exploration into an approximation strategy, designed to be as close to the optimal width for as many Dyck words as possible.
    \item An analysis of random strategies to compare how a strategy does against the average.
    \item Potential Dyck word generation functions to test strategies against pseudo-random inputs.
    \item Speeding up the exhaustive width search process, via memoization or 
\end{enumerate}
