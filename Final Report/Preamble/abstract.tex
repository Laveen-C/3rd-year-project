\clearpage
\thispagestyle{empty}
\nointerlineskip
\vspace*{\fill}
    
\section*{\center \abstractname}
The project expands on the work of Chistikov and Vyali, which introduced a simple one-player game; The re-pairing game can be played on any well-formed sequence of opening and closing brackets (a Dyck word). A move consists of "pairing" any opening bracket with any closing bracket to the right of it, and "erasing" the two. The process is repeated until we are left with no remaining brackets. Such a game can have many strategies, but the effectiveness of a strategy is measured by it's width, which is the maximum number of nonempty segments of symbols seen during a play of the game.

\textbf{Keywords:} \textit{Dyck language, Re-pairing brackets, Combinatorics, Web application, Python, ReactJS}

\section*{\center Acknowledgements}
I'd like to thank my dissertation supervisors, Dr. Dmitry Chistikov and Dr. Matthias Englert, for their invaluable guidance, support and feedback throughout this project. 

I'd also like to thank Melany Henot, Varshneyan Prabakaran, Brendan Bell, Raumaan Ahmed, Varun Chodanker and Devon Connor for engaging in insightful discussions on some of the problems I tackled, and for their continued support which enabled me to complete this project throughout a difficult year.

\vspace*{\fill}
