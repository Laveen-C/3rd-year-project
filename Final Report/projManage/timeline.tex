\section{Timeline}

The project begun shortly after the start of term 1. Most of term 1 was spent reviewing the literature and attempting to absorb as much as possible. During this period, the goal was to try and conduct a ``pen and paper'' analysis of the re-pairing game, becoming as familiar with it as possible. Many connections and observations were made independently, but all of them turned out to be well known in some way or another. 

Towards the end of term 1, the project shifted to a more software oriented approach, and so the goals were reworked with with this in mind. Through discussions with the supervisor, the author decided to create a software to visualise and demonstrate the game, but to also have the potential for analysis and exhaustive purposes. The last few weeks of term 1 were spent experimenting with different concepts for software. A local Python application was tested, but after some deliberation the idea of developing a web application approach was finalised. 

Over the Christmas holidays, time was allocated to writing up the first half of the report (the literature review portion) and becoming familiar with the libraries and frameworks to be used. Due to the author's unfamiliarity with web development, more time was spent on learning the relevant technologies than anticipated. 

At the beginning of term 2, skeleton code for the website had been written, and majority of the content for the literature review was put down into the report. Much of term 2 was spent building upon and refining the existing codebase for the web application. The simple and greedy strategies had already been implemented and tested successfully, and as the author was becoming much more familiar with React, the frontend started to become more intricate with multiple approaches for manual re-pairing being tested around this time. 

Due to health issues progress had then stagnated, but swiftly resumed during the next academic year in term 1. Here, the author spent time refining the proofs and observations made in the report to make them more rigorous. The author performed a deeper study into asynchronous functions, the misunderstanding of which caused many errors during the development process, to understand where many of the unwanted functionality was coming from. This proved to be an important step for the development of the web application, since all of the communication between the backend and the frontend required the usage of asynchronous functions.

During term 2, the author spent much time researching into methods of speeding up the exhaustive width search. The \hyperref[claim:samePrime]{claim on Dyck primes} made a significant speedup, and the discovery of PyPy was an additional resource to leverage. Around this time the writing for the software development portion of the project begun. 

The majority of time beyond this point was spent finishing the report, polishing the codebase of the web application and analysing edge cases.