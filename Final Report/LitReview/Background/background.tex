\section{Background}
Before we delve into the re-pairing game itself, it's important to understand the objects that make up the game itself. We provide a brief but rigorous background on ideas relevant to this game below.

\subsection{Dyck Paths}
We start with the concept of Dyck paths. These are paths of length $2n$ where the path starts at $(0,0)$ and finishes at $(2n,0)$, the only legal moves are $(1,1)$ or $(1,-1)$, and the path does not cross below the x-axis at any point. More formally:

\begin{definition}
    A \textbf{Dyck Path} is a sequence of vectors $(a,b)$ starting at $(0,0)$ and finishing at $(2n,0)$, such that every step on the walk goes from some $(a,b)$ to either $(a+1,b+1)$ or $(a+1,b-1)$.
\end{definition}

Here's an example of such a path:

\begin{figure}[H]
    \centering
    \includestandalone{./figures/dyckPathEx/dyckPathEx}
    \caption{An example Dyck path}
    \label{fig:dyckPath}
\end{figure}
The move $(1,1)$ is called a "rise", and the move $(1,-1)$ a fall. Notice that all Dyck paths must have more rises than falls at any point during the path to stay above the x-axis, and to reach $(2n,0)$ we must have had as many rises as falls in the whole sequence. 

We can describe a Dyck path as a string of opening and closing brackets, where \texttt{(} corresponds to a rise, and \texttt{)} corresponds to a fall. We call such strings Dyck words.

\subsection{Dyck Words}
\begin{definition}
    A \textbf{Dyck word} is a string that satisfies the following:
    \begin{enumerate}
        \item The only characters in the string are ``\texttt{(}'' or ``\texttt{)}''.
        \item At any point in the string, there are no more ``\texttt{(}'' brackets than ``\texttt{)}'' brackets.
        \item The string contains the same number of ``\texttt{(}'' and ``\texttt{)}'' brackets
    \end{enumerate}
\end{definition}
It should be clear that every Dyck word corresponds to a unique Dyck path.

Alternatively, we can define a Dyck word through the convention ``\texttt{(}'' $= 1$ and ``\texttt{)}'' $= -1$:

\begin{definition}[{\cite{chistikov2020re}}]
    \label{def:seqDyck}
    A \textbf{Dyck word} is a sequence $\sigma=(\sigma(1), \dots, \sigma(n))_{n\geq 1}$ such that for all $1\leq i\leq n$, we have ${\sigma(i)}\in\{1, -1\}$ and the following is satisfied:
    \begin{enumerate}
        \item For every \! $1 \leq k \leq n$, we have \! $\sum_{i = 1}^{k} \sigma(i) \geq 0$ 
        \item $\sum_{i = 1}^{n} \sigma(i) = 0$
    \end{enumerate} 
\end{definition}

These definitions are equivalent, and will be used interchangeably throughout this report.

\subsection{North-East Lattice Walks}
These words have interesting combinatorial properties, one of which is their unsurprising link to Lattice Walks:

\begin{definition} 
    A \textbf{North-East lattice walk} is a sequence of vectors $(a, b)$ starting at $(0, 0)$ and finishing at $(n, n)$, such that every step on the walk goes from some $(a, b)$ to either $(a+1, b)$ or $(a, b+1)$.
\end{definition}
In other words, every step is exactly one unit vector right or one unit vector up.

\begin{theorem}
    Every Dyck word of length $2n$ corresponds to a unique North-East lattice walk from $(0, 0)$ to $(n, n)$ that does not cross above the line $y = x$. 
\end{theorem}

\begin{figure}[h]
    \centering
    \includestandalone{./figures/latticeWalkEx/latticeWalkEx}
    \caption{An example North-East lattice walk for \texttt{(()())(())}}
    \label{fig:latticeWalk}
\end{figure}

This is a trivial mapping, as we can take a move east $(1,0)$ to be ``\texttt{(}'', and a move north $(0,1)$ to be ``\texttt{)}''. This works out as expected since we have $2n$ brackets to match, and $n$ moves east and north each during such lattice walks. Our restriction on Dyck words that ensures there are never more ``\texttt{)}'' brackets than ``\texttt{(}'' brackets corresponds to the restriction of the walk being below (or on) the line $y = x$. This connection gives us another result:

\begin{theorem}[{\cite{rukavicka2011dp}}]
    The number of Dyck words of length $2n$ corresponds to the n\textsuperscript{th} Catalan number $C_{n}=\sum_{i=1}^{n-1}C_{i}C_{n-1-i}$.
\end{theorem}
\begin{proof}
    Let $W_{n}$ be the number of North-East lattice walks from $(0,0)$ to $(n,n)$ that do not cross above the line $y = x$. Given any such walk, it may touch this line at any $(i, i)$ where $0 \leq i \leq n-1$. This gives us a natural recursion; every such North-East lattice walk can be broken down into a North-East lattice walk from $(0, 0)$ to $(i, i)$, and from $(i, i)$ to $(n, n)$. Using this idea, we see that $W_{n}=\sum_{i=0}^{n-1}W_{i}W_{n-1-i}$. This is precisely the recursion for the $n\textsuperscript{th}$ Catalan number.
\end{proof}

\subsection{Binary Trees}
However, we can also relate this result to the concept of binary trees as well. Suppose we had four factors of a number $x = x_{1}x_{2}x_{3}x_{4}$. How many ways can we \textit{completely parenthesize} this expression of factors (i.e. place brackets around this to group factors)?
\begin{enumerate}
    \item $(x_{1}(x_{2}x_{3}))x_{4}$
    \item $((x_{1}x_{2})x_{3})x_{4}$
    \item $x_{1}(x_{2}(x_{3}x_{4}))$
    \item $x_{1}((x_{2}x_{3})x_{4})$
    \item $(x_{1}x_{2})(x_{3}x_{4})$
\end{enumerate}
There are 5 ways, but this is the same as $C_3 = \frac{1}{3+1} {2(3) \choose 3} = 5$. To understand why, let $G_{n}$ be the number of ways to place brackets around $x = x_{1}\dots x_{n}$. Note that the outermost factor will not be within any pair of brackets.

We look at the leftmost ``\texttt{(}'', and its matching ``\texttt{)}''. Suppose this surrounds $0 \leq i \leq n-1$ factors. We can completely parenthesize this expression of factors in $G_{i}$ ways. Since the outermost factor will not be within any pair of brackets, this leaves a further $n-i-1$ factors to completely parenthesize, which can be done in $G_{n-i-1}$ ways. Therefore the number of ways to completely parenthesize an expression of $n$ factors is ${G_{n}=\sum_{i=0}^{n-1}G_{i}G_{n-1-i}}$, which is precisely the $n\textsuperscript{th}$ Catalan number.
\begin{theorem}
    The number of proper binary trees with $n+1$ leaves is $C_n$.
\end{theorem}
\begin{proof}
    The idea is to show that every proper binary tree of $n+1$ leaves corresponds to a unique completely parenthesized expression of $n$ factors. Recall that a proper binary tree is a tree such that every non-leaf node (internal node) has 2 children.
    
    Take any completely parenthesized expression of $x~=~x_{1}\dots~x_{n+1}$. Then, we can derive a proper binary tree from this expression as follows. Take the innermost pair of brackets, and the factors they contain $x_ix_{i+1}$. Let these be leaf nodes, going to some internal $y_1 = x_ix_{i+1}$. Then take the next innermost pair of brackets, which will contain $y_1x_j$ with $j\neq i,i+1$. Then let these be the children of the node $y_2$. This process repeats all the way up until we get to $x_1$. 
    
    It should be clear that every factor of $x$ will be a leaf node. If two expressions give the same tree, by our construction it is also clear that they must be the same expression. As we've previously shown that the number of ways to completely parenthesize an expression of $n+1$ variables is $C_n$, by this bijection the number of ways to construct a proper binary tree with $n+1$ leaves is also $C_n$.
\end{proof}

\begin{figure}[H]
    \centering
    \includestandalone{./figures/completeBinTree/completeBinTree}
    \caption{The complete binary tree corresponding to the parenthesized expression $x_1(x_2(x_3x_4))$}
    \label{fig:completeBinTree}
\end{figure}



