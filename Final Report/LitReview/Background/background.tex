\newtheorem{definition}{Definition}
\newtheorem{theorem}{Theorem}

\section{Background}

\subsection{Dyck Words}
\begin{definition}
    A \textbf{Dyck Word} is a string that satisfies the following:
    \begin{enumerate}
        \item The only characters in the string are ``\texttt{(}'' or ``\texttt{)}''.
        \item At any point in the string, there are no more ``\texttt{(}'' brackets than ``\texttt{)}'' brackets.
        \item The string contains the same number of ``\texttt{(}'' and ``\texttt{)}'' brackets
    \end{enumerate}
\end{definition}

\noindent We can alternatively define a Dyck word through the convention ``\texttt{(}'' $= 1$ and ``\texttt{)}'' $= -1$:

\begin{definition}[{\cite{chistikov2020re}}]
    A \textbf{Dyck word} is a sequence $D=(d_{n})_{n\geq 1}$ such that ${d_{n}\in\{1, -1\}}$ and the following is satisfied:
    \begin{enumerate}
        \item For every \! $1 \leq k \leq n$, we have \! $\sum_{i = 1}^{k} d_{i} \geq 0$ 
        \item $\sum_{i = 0}^{n} d_{i} = 0$
    \end{enumerate} 
\end{definition}

\subsection{North-East Lattice Walks}
\noindent These words have interesting combinatorial properties, one of which is their unsurprising link to Lattice Walks:

\begin{definition}
    \label{seqDyck}
    A \textbf{North-East lattice walk} is a sequence of vectors $(a, b)$ starting at $(0, 0)$ and finishing at $(n, n)$, such that every step on the walk goes from some $(a, b)$ to either $(a+1, b)$ or $(a, b+1)$.
\end{definition}
\noindent In other words, every step is exactly one unit vector right or one unit vector up.

\begin{theorem}
    Every Dyck word of length $2n$ corresponds to a unique North-East lattice walk from $(0, 0)$ to $(n, n)$ that does not cross above the line $y = x$. 
\end{theorem}

\begin{figure}
    \centering
    \includestandalone{./figures/latticeWalkEx/latticeWalkEx}
    \caption{North-East lattice walk for \texttt{(()())(())}}
    \label{fig:dyck_path}
\end{figure}

\noindent This is a trivial mapping, as we can take a move east $(1,0)$ to be ``\texttt{(}'', and a move north $(0,1)$ to be ``\texttt{)}''. This works out as expected since we have $2n$ brackets to match, and $n$ moves east and north each during such lattice walks. Our restriction on Dyck words that ensures there are never more ``\texttt{)}'' brackets than ``\texttt{(}'' brackets corresponds to the restriction of the walk being below (or on) the line $y = x$. This connection gives us another result:

\begin{theorem}[{\cite{rukavicka2011dp}}]
    The number of Dyck words of length $2n$ corresponds to the n\textsuperscript{th} Catalan number $C_{n}=\sum_{i=1}^{n-1}C_{i}C_{n-1-i}$.
\end{theorem}
\begin{proof}
    Let $W_{n}$ be the number of North-East lattice walks from $(0,0)$ to $(n,n)$ that do not cross above the line $y = x$. Given any such walk, it may touch this line at any $(i, i)$ where $0 \leq i \leq n-1$. This gives us a natural recursion; every such North-East lattice walk can be broken down into a North-East lattice walk from $(0, 0)$ to $(i, i)$, and from $(i, i)$ to $(n, n)$. Using this idea, we see that $W_{n}=\sum_{i=0}^{n-1}W_{i}W_{n-1-i}$. This is precisely the recursion for the $n\textsuperscript{th}$ Catalan number.
\end{proof}

\subsection{Binary Trees}
\noindent However, we can also relate this result to the concept of binary trees as well. Suppose we had four factors of a number $x = x_{1}x_{2}x_{3}x_{4}$. How many ways can we \textit{completely parenthesize} this expression of factors (i.e. place brackets around this to group factors)?
\begin{enumerate}
    \item $(x_{1}(x_{2}x_{3}))x_{4}$
    \item $((x_{1}x_{2})x_{3})x_{4}$
    \item $x_{1}(x_{2}(x_{3}x_{4}))$
    \item $x_{1}((x_{2}x_{3})x_{4})$
    \item $(x_{1}x_{2})(x_{3}x_{4})$
\end{enumerate}
There are 5 ways, but this is the same as $C_3 = \frac{1}{3+1} {2(3) \choose 3} = 5$. To understand why, let $G_{n}$ be the number of ways to place brackets around $x = x_{1}\dots x_{n}$. Note that the outermost factor will not be within any pair of brackets.
\par\null
\noindent We look at the leftmost ``\texttt{(}'', and its matching ``\texttt{)}''. Suppose this surrounds $0 \leq i \leq n-1$ factors. We can completely parenthesize this expression of factors in $G_{i}$ ways. Since the outermost factor will not be within any pair of brackets, this leaves a further $n-i-1$ factors to completely parenthesize, which can be done in $G_{n-i-1}$ ways. Therefore the number of ways to completely parenthesize an expression of $n$ factors is ${G_{n}=\sum_{i=0}^{n-1}G_{i}G_{n-1-i}}$, which is precisely the $n\textsuperscript{th}$ Catalan number.
\begin{theorem}
    The number of proper binary trees with $n+1$ leaves is $C_n$.
\end{theorem}
\begin{proof}
    The idea is to show that every proper binary tree of $n+1$ leaves corresponds to a unique completely parenthesized expression of $n$ factors. Recall that a proper binary tree is a tree such that every non-leaf node has 2 nodes. 
\end{proof}