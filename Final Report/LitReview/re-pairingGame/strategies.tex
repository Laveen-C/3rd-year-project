\section{Re-Pairing Strategies}

We can now begin to discuss strategies for playing the re-pairing game. By strategy, we refer to a set of rules that we can follow to generate a sequence of moves which will always yield $\varepsilon_n$. We'll describe all strategies from a high level, and then give algorithms for running the strategies on a given input. All algorithms will use the sequential definition of a Dyck word.

\subsection{Simple}
We start with the most simple strategy. Given any Dyck word, we already have a naturally existing pairing, namely the pairs of matching brackets. However, given any simple re-pairing, we do not have a clear order on which matching brackets to pair first, and we have seen that the order matters greatly when attempting to minimise the width.

We'll begin with the most natural ordering, namely pairing each leftmost bracket with its matching right bracket. This means for two moves $(l_i, r_i)$ and $(l_j, r_j)$, if $l_i < l_j$ then we perform $(l_i, r_i)$ first.

\begin{algorithm}
    \caption{Leftmost Simple Re-Pairing}
    \begin{algorithmic}
        \State $c \gets 0$
        \State $p \gets 0$
    \end{algorithmic}
\end{algorithm}