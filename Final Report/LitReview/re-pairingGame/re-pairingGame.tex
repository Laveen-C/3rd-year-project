\section{The Re-Pairing Game}
Now that we've formalised some of the background knowledge, we can start to discuss the re-pairing game. 

\subsection{Re-Pairings}
\begin{definition}[\cite{chistikov2020re}]
    Formally, a \textbf{re-pairing} of a Dyck word of length n is a sequence of ordered pairs of indices $m = (m_{1}, \dots, m_{\frac{n}{2}})$, where each $m_{i} = (l_{i}, r_{i})$ for $l_{i}, r_{i} \in \{1,\dots,n\}$, and the following properties are satisfied:
    \begin{enumerate}
        \item For all $1\leq i\leq \frac{n}{2}$, we have $l_{i} \leq r_{i}$, $\sigma(l_{i}) = 1$ and $\sigma(r_{i} = -1)$. 
        \item Indices $l_{i}$ and $r_{i}$ appear in no other ordered pair $m_{j}$ for $j \neq i$.
    \end{enumerate}
\end{definition}

This says that a re-pairing consists of ordered pairs of indices of the Dyck word, where each pair consists of an index for a left bracket and an index for a right bracket to its right, and these indices do not appear in any other ordered pair.

Since a Dyck word of length $n$ must be even, we must have $\frac{n}{2}$ ordered pairs, and it is clear this will be a positive integer.

To play the game, we define any valid re-pairing $m$ on a valid Dyck word, and start at the ordered pair $m_{1}$. We take this pair, take the relevant 

Similarly, we can also look at re-pairings for a word that has been \textbf{partially re-paired}, meaning the re-pairing has not completely erased the word. We call these \textbf{Partial Dyck Words}. 

\subsection{Height}
Define height. Link to strahler numbers?

\subsection{Width}
Define width. Link to ?

\subsection{Observation}
We start with an observation
All observations e.g. no pairing outside of components allowed

\subsection{Strategies}
Start listing strategies, then their algorithms. Simple, Non-Simple, Greedy. Talk about a brute-force idea!

\subsection{Factors and Dyck Primes}
